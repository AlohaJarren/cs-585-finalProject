\documentclass[11pt]{article}

\pdfoutput=1

% --------------------------------------------------------------------------- %
% Packages:
% --------------------------------------------------------------------------- %

\usepackage{amsmath,amsthm,amssymb,mathtools} %
\usepackage{authblk} %
\usepackage[english]{babel} %
\usepackage{cite} % references in numerical order
\usepackage{color} %
\usepackage{csquotes} %
\usepackage{dsfont} %
\usepackage[breaklinks]{hyperref} %
\usepackage[hmargin=1in,vmargin=1in]{geometry} %
\usepackage{graphicx} %
\usepackage{hyphenat} %
\usepackage{mathpazo} % charter, fourier, mathpazo, times
\usepackage{mdframed} %
\usepackage{suffix} % for *-version commands
\usepackage{tikz} %
\usepackage{xspace} %
\usepackage{enumitem} % Added to handle notation section

\frenchspacing

\hypersetup{colorlinks=true, linkcolor=blue, citecolor=magenta}

\usetikzlibrary{calc,positioning}

\mdfdefinestyle{figstyle}{ %
  linecolor=black!7, %
  backgroundcolor=black!7, %
  innertopmargin=10pt, %
  innerleftmargin=25pt, %
  innerrightmargin=25pt, %
  innerbottommargin=10pt %
}

\definecolor{White}{rgb}{1,1,1} %
\definecolor{Black}{rgb}{0,0,0} %
\definecolor{LightGray}{rgb}{.8,.8,.8} %
\colorlet{ChannelColor}{LightGray} %
\colorlet{ChannelTextColor}{Black} %
\colorlet{ReadoutColor}{White} %

% --------------------------------------------------------------------------- %
% Macros:
% --------------------------------------------------------------------------- %

\newtheorem{theorem}{Theorem} %
\newtheorem{lemma}[theorem]{Lemma} %
\newtheorem{proposition}[theorem]{Proposition} %
\newtheorem{corollary}[theorem]{Corollary} %
\theoremstyle{definition} %
\newtheorem{definition}[theorem]{Definition} %
\theoremstyle{remark} %
\newtheorem{remark}{Remark} %

\newcommand{\comment}[1]{\begin{quote}\sf
    \textcolor{blue}{[#1]}\end{quote}} %

\newcommand{\complex}{\mathbb{C}} %
\newcommand{\real}{\mathbb{R}} %
\renewcommand{\natural}{\mathbb{N}} %
\renewcommand{\Re}{\operatorname{Re}} %
\renewcommand{\Im}{\operatorname{Im}} %
\newcommand{\secpar}{\kappa} %
\newcommand{\secparam}{1^\secpar} %
\newcommand{\poly}{\mathit{poly}} %
\def\yes{\text{yes}} %
\def\no{\text{no}} %

\newenvironment{mylist}[1]{\begin{list}{}{ %
      \setlength{\leftmargin}{#1} %
      \setlength{\rightmargin}{0mm} %
      \setlength{\labelsep}{2mm} %
      \setlength{\labelwidth}{8mm} %
      \setlength{\itemsep}{0mm}}}{\end{list}}

%% End-Of-Header

% --------------------------------------------------------------------------- %
% Your customization:
% --------------------------------------------------------------------------- %
\def\course{[\textsc{Fall 2025 CS485/585}]}

\newcommand{\notationdef}[2]{%
  \par\noindent
  \hangindent=2em
  \hangafter=1
  \textit{#1} #2\par\vspace{0.5em}
}

% --------------------------------------------------------------------------- %
% Main document:
% --------------------------------------------------------------------------- %

\begin{document}

\title{\course \\
  \LARGE\bf {Project report: Differential Cryptanalysis against 1-Round DES}}

\author[ ]{Jarren Calizo}
\author[ ]{Benjamin Chong}
\author[ ]{Wesley Grzemkowski}

\affil[ ]{Department of Computer Science\protect\\
  Portland State University\vspace{2mm}}
\affil[ ]{\textit {\{calizo,chongben,wesleyg\}@pdx.edu}}

\date{\today}

\renewcommand\Affilfont{\normalsize\itshape}
\renewcommand\Authfont{\large}
\setlength{\affilsep}{6mm}
\renewcommand\Authsep{\rule{10mm}{0mm}}
\renewcommand\Authands{\rule{10mm}{0mm}}

\maketitle
\setcounter{page}{0}

\thispagestyle{empty}

\begin{abstract}

  An executive summary of your report.


  \vfill
  Your report should be about 10 pages without counting this title
  page or the references in the end. Give a comprehensive account but
  concisely. 
\end{abstract}

\newpage

%------------------------------------------------------------------------------%
\section{Introduction}
\label{sec:intro}
%------------------------------------------------------------------------------%

        DES is a 16-round encryption scheme that operates on a 64-bit message using a 56-bit key.
    The full 16-round DES scheme has proven rather secure considering its limited key space. It has however fallen out of use since the relatively small key space makes the scheme vulnerable to brute-force attacks. From what we have found in our research, reducing the number of rounds DES uses makes it vulnerable to differential cryptanalysis attacks.
    
        In this report, we will show the attack we implemented to recover the key for a one-round 
    DES scheme. We will also show how such an attack can be used to recover the key for DES with a higher round count with relative efficiency.

\subsubsection*{Organization} The remainder of the paper is organized
as follows: $\ldots$

%------------------------------------------------------------------------------%
\section{Preliminaries}
\label{sec:prelim}
%------------------------------------------------------------------------------%
This section provides preliminary materials, such as summarizing
common notations, definitions, and useful facts for the main body.

\subsection{Notation}
We use the following notations in the discussion of our attack.

\begin{description}[style=unboxed, leftmargin=2em, font=\itshape]
    \item[Numbers:] To differentiate between hexadecimal, decimal, and binary number a subscript is used to denote the number's base. Decimal numbers are denoted with no subscript. (e.g., $31 =\text{1F}_{16} = 00011111_{2}$)
    
    \item[Plaintext and Ciphertext:] The plaintext is denoted by $M$, and the ciphertext is denoted by $C$. Because the initial permutation has little bearing on our attack model, $M$ represents the plaintext after the initial permutation, and $C$ represents the ciphertext before the final permutation.

    \item[Feistel Rounds:] The state of the ciphertext after round is $i$ denoted by the pair $(L_i, R_i)$, where $L_i$ denotes the leftmost 32-bits and $R_i$ denotes the rightmost 32-bits. The pair $(L_0, R_0)$ is used to denote the input to the first round of the feistel network.

    \item[Differences:] In differential cryptanalysis, variables are usually considered in pairs. Given a value $X$, its corresponding value in the pair is denoted by $X'$. The difference of $X$ and $X'$ is defined as $X \oplus X'$, and this value is denoted by $\Delta X$.

    \item[Subkeys:] The subkeys is denoted by $K_i$, where $i$ indicates the round in which the subkey is used. The notation $K^j$ is used to denote the $j$-th 6-bit chunk of $K$.

    \item[Initial Permutation:] The initial permutation of DES is denoted by $IP(X)$, and the inverse initial permutation, or final permutation, is denoted by $IP^{-1}(X)$.
    
    \item[F Function:] The F function, or mangler function, is denoted by $F(X)$.
    
    \item[Expansion Function:] The expansion function is denoted by $E(X)$.

    \item[P Permutation:] The P permutation is denoted by $P(X)$.

    \item[S-boxes:] The collective 48-bit to 32-bit S-box layer is denoted by the function $S(X)$. The individual eight S-boxes are denoted by $S^1, S^2, \dots,S^8$.

    \item[Intermediary Values:] The output of the expansion function is denoted by $B$. The input to the S-box layer is denoted by $I$. The output of the S-box layer is denoted by $O$. A superscript $j$ is used to denote the $j$-th 6-bit chunk, or 4-bit chunk.
    
\end{description}

\subsection{1-round DES}

We chose to construct our attack against 1-round DES. It is important to first clarify that 1-Round DES is already a trivial algorithm to defeat. To show this, consider a single known plaintext-ciphertext pair $M =(L_0, R_0)$ and $C=(L_1, R_1)$. From this values, one can compute $B = E(R_0)$ and $O = P^{-1}(R_1 \oplus L_0)$.
%------------------------------------------------------------------------------%
\section{DES Overview}
%------------------------------------------------------------------------------%

%------------------------------------------------------------------------------%
\subsection{Key Generation}
%------------------------------------------------------------------------------%

    Each of the 16 rounds of DES uses its own 48-bit subkey. Each of these subkeys are obtained by 
inputting a single 64-bit key into the key generation algorithm. Note that the true keyspace of DES is 56 bits. When a 64-bit key is given as input, the 8th bit of each byte is not used.
    The 56 bits that are used are then passed through a permutation. This permutation is the same
for all keys. The left half of the output of this permutation becomes $C_0$ and the right half becomes $D_0$. We then use a series of left shifts on $C_0$ and $D_0$ to obtain $C_n$ and $D_n$ for $1<=n<=16$. for $n = 1, 2, 9, 16$ $C_n$ and $D_n$ are a single left shift from $C_{n-1}$ and $D_{n-1}$, and for the other values of n, two left shifts are used. We then have sixteen 56-bit strings $C_nD_n$ for $1<=n<=16$. To obtain our 48-bit subkeys $K_n$ we apply a permutation to 48 bits of the corresponding $C_nD_n$. For reduced round DES of n rounds, we can just use the first n keys.
%------------------------------------------------------------------------------%
\subsection{Encryption Overview}
%------------------------------------------------------------------------------%

To encrypt a 64 bit message under DES, the message is first passed through a set initial permutation $IP$ that rearranges the bits. This permutation is the same for every encryption. The output of this permutation becomes the input to the first round. The output of that round becomes the input for the second round. This continues until the output of the sixteenth round, which is run through the inverse of $IP$. The output of that inverse permutation is the ciphertext.

For each of the rounds, the input is split into a right half $R$ and a left half $L$ each a 32-bit string. Each round uses the function $F_K$ where $K$ is the subkey for that round. The left half of the rounds output is $F_K(R)\oplus L$ and the right half is simply $L$.
%------------------------------------------------------------------------------%
\subsection{F Function}
%------------------------------------------------------------------------------%

The F function, also called the mangler function, is designed to produce a pseudorandom 32-bit output from a 32-bit input and 48-bit key. The input to $F$ is runthrough the expansion box $E$, producing a 48-bit string, which is then XOR-ed with the key. The output of the XOR is split into 8 6-bit strings. These strings become the input for the 8 S-boxes. Each S-box is a unique mapping of 6-bit strings to 4-bit strings. After the S-box layer, we combine the eight resulting 4-bit strings into a 32-bit string and use the P permutation to produce the 32-bit output of the F function.

%------------------------------------------------------------------------------%
\subsection{Vulnerabilities}
%------------------------------------------------------------------------------%
Since The expansion box and P permutation in the F function are predetermined, much of the strength of this scheme relies on the S boxes. For our attack we focus on the XORs of pairs of messages, and the corresponding XORs of the pairs of ciphertexts. The problem with the S boxes is that for a given input XOR, some output XORs are more likely to coour than others. We can use this to narrow down the possible keys by choosing certain plaintext pairs with certain XORs.
%------------------------------------------------------------------------------%
\section{The Attack}
\label{sec:result}
%------------------------------------------------------------------------------%

Pick appropriate titles and subtitles according to the topic of your
project.

%------------------------------------------------------------------------------%
\subsection{Our Attack Model}
\label{sec:result1}
%------------------------------------------------------------------------------%

The attack we have created is a chosen-plaintext attack against 1-Round DES. However it could also be adapted to be a known-plaintext attack. This attack is not intended to be an effective or practical attack on single-round DES, but rather demonstrates how DES may be susceptible to differential cryptanalysis using a simplified model.

The goal of this attack is to generate a number ciphertext pairs where the following information is known: the difference in the plaintexts, the difference in the ciphertexts, and the input to the F function.

%------------------------------------------------------------------------------%
\subsection{Finding Characteristics}
\label{sec:result2}
%------------------------------------------------------------------------------%


%------------------------------------------------------------------------------%
\subsection{Generating Plaintext Ciphertext Pairs}
\label{sec:result2}
%------------------------------------------------------------------------------%


%------------------------------------------------------------------------------%
\subsection{Reducing Key Space}
\label{sec:result2}
%------------------------------------------------------------------------------%


%------------------------------------------------------------------------------%
\subsection{Brute-forcing Subkey}
\label{sec:result2}
%------------------------------------------------------------------------------%

%------------------------------------------------------------------------------%
\section{Something more}
\label{sec:more}
%------------------------------------------------------------------------------%


%-----------------------------------------------------------------------------%
\section{Conclusion}
\label{sec:con}
%-----------------------------------------------------------------------------%

This project implemented and experimentally evaluated a differential cryptanalysis attack on a one-round instance of DES, with the goal of recovering the last-round subkey from chosen plaintext-ciphertext pairs. By constructing S-box difference distribution tables, generating structured plaintext pairs with fixed input differences, and aggregating votes over candidate subkeys, the attack consistently reduced the 48-bit last-round key space from 2^48 possibilities down to roughly 2^10-2^12 candidates, after which a small brute-force search recovered the correct subkey. Although 1-round DES is algebraically trivial to invert given a single known plaintext-ciphertext pair, treating it as a target for differential cryptanalysis provided a concrete, code-level view of how biased differentials arise and how they can be exploited.

Our findings align qualitatively with the classical literature. Biham and Shamir's original work showed that full 16-round DES admits differential characteristics with non-negligible bias, but exploiting them requires about 2^47 chosen plaintexts and substantial time, pushing such attacks beyond realistic deployment scenarios even though they are asymptotically faster than exhaustive key search. Matsui's linear cryptanalysis paints a similar picture from a different angle: linear approximations can be used to recover DES keys with around 2^43-2^47 known plaintexts, again demonstrating structural weaknesses that are mostly of academic interest. Our 1-round experiments can be viewed as a scaled-down analogue of these results, where the same differential machinery becomes visible with far fewer pairs and the implementation details remain tractable for our cryptography class's final project.

At the same time, the limitations of our model are significant. We attack only the final round, not the full 16-round cipher; we assume ideal chosen-plaintext access to the encryption oracle; and we do not attempt to propagate differentials across multiple rounds or invert the full key schedule to recover the 56-bit master key. In practice, modern attackers would simply brute force single-DES or target Triple DES and legacy protocols, while contemporary standards have moved to AES and other ciphers with larger keys and stronger security margins. Our contribution is therefore best understood as a educational case study rather than a new break.

Nevertheless, the exercise is valuable. Implementing DES, building DDTs, and debugging the attack sharpened our understanding of how S-boxes, permutations, and key mixing interact under differential analysis. The fact that meaningful key-space reductions are achievable even in this simple setting reinforces why block cipher designers must select S-boxes and round counts with respect to differential and linear criteria, not just intuition. Future work could extend this framework to multiple rounds, compare DES's S-boxes against randomly generated alternatives, or replicate the experiment on modern lightweight ciphers to contrast their resistance profiles with DES.


%-----------------------------------------------------------------------------%
\subsection*{Acknowledgments}
%-----------------------------------------------------------------------------%

Optional. 

\newpage

\appendix

%------------------------------------------------------------------------------%
\section{Some supplementary material}
\label{sec:appendix}
%------------------------------------------------------------------------------%

Something you think is worth mentioning but not essential or digresses
the flow of the main body, e.g., long proofs of some claims.  Use it
scarcely. (Is it really worth including? Unless 100\% certain, the
answer is most likely no.)

% --------------------------------------------------------------------------- %
% References
% Check overleaf tutorial on managing refs with bibtex \url{https://www.overleaf.com/learn/latex/Bibliography_management_with_bibtex}
% --------------------------------------------------------------------------- %
\newpage
\bibliographystyle{alpha}
% \bibliographystyle{acm}

\bibliography{project.bib}

\end{document}

% LocalWords:  transformative

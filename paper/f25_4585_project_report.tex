\documentclass[11pt]{article}

\pdfoutput=1

% --------------------------------------------------------------------------- %
% Packages:
% --------------------------------------------------------------------------- %

\usepackage{amsmath,amsthm,amssymb,mathtools} %
\usepackage{authblk} %
\usepackage[english]{babel} %
\usepackage{cite} % references in numerical order
\usepackage{color} %
\usepackage{csquotes} %
\usepackage{dsfont} %
\usepackage[breaklinks]{hyperref} %
\usepackage[hmargin=1in,vmargin=1in]{geometry} %
\usepackage{graphicx} %
\usepackage{hyphenat} %
\usepackage{mathpazo} % charter, fourier, mathpazo, times
\usepackage{mdframed} %
\usepackage{suffix} % for *-version commands
\usepackage{tikz} %
\usepackage{xspace} %
\usepackage{enumitem} % Added to handle notation section

\frenchspacing

\hypersetup{colorlinks=true, linkcolor=blue, citecolor=magenta}

\usetikzlibrary{calc,positioning}

\mdfdefinestyle{figstyle}{ %
  linecolor=black!7, %
  backgroundcolor=black!7, %
  innertopmargin=10pt, %
  innerleftmargin=25pt, %
  innerrightmargin=25pt, %
  innerbottommargin=10pt %
}

\definecolor{White}{rgb}{1,1,1} %
\definecolor{Black}{rgb}{0,0,0} %
\definecolor{LightGray}{rgb}{.8,.8,.8} %
\colorlet{ChannelColor}{LightGray} %
\colorlet{ChannelTextColor}{Black} %
\colorlet{ReadoutColor}{White} %

% --------------------------------------------------------------------------- %
% Macros:
% --------------------------------------------------------------------------- %

\newtheorem{theorem}{Theorem} %
\newtheorem{lemma}[theorem]{Lemma} %
\newtheorem{proposition}[theorem]{Proposition} %
\newtheorem{corollary}[theorem]{Corollary} %
\theoremstyle{definition} %
\newtheorem{definition}[theorem]{Definition} %
\theoremstyle{remark} %
\newtheorem{remark}{Remark} %

\newcommand{\comment}[1]{\begin{quote}\sf
    \textcolor{blue}{[#1]}\end{quote}} %

\newcommand{\complex}{\mathbb{C}} %
\newcommand{\real}{\mathbb{R}} %
\renewcommand{\natural}{\mathbb{N}} %
\renewcommand{\Re}{\operatorname{Re}} %
\renewcommand{\Im}{\operatorname{Im}} %
\newcommand{\secpar}{\kappa} %
\newcommand{\secparam}{1^\secpar} %
\newcommand{\poly}{\mathit{poly}} %
\def\yes{\text{yes}} %
\def\no{\text{no}} %

\newenvironment{mylist}[1]{\begin{list}{}{ %
      \setlength{\leftmargin}{#1} %
      \setlength{\rightmargin}{0mm} %
      \setlength{\labelsep}{2mm} %
      \setlength{\labelwidth}{8mm} %
      \setlength{\itemsep}{0mm}}}{\end{list}}

%% End-Of-Header

% --------------------------------------------------------------------------- %
% Your customization:
% --------------------------------------------------------------------------- %
\def\course{[\textsc{Fall 2025 CS485/585}]}

\newcommand{\notationdef}[2]{%
  \par\noindent
  \hangindent=2em
  \hangafter=1
  \textit{#1} #2\par\vspace{0.5em}
}

% --------------------------------------------------------------------------- %
% Main document:
% --------------------------------------------------------------------------- %

\begin{document}

\title{\course \\
  \LARGE\bf {Project report: Differential Cryptanalysis against 1-Round DES}}

\author[ ]{Jarren Calizo}
\author[ ]{Benjamin Chong}
\author[ ]{Wesley Grzemkowski}

\affil[ ]{Department of Computer Science\protect\\
  Portland State University\vspace{2mm}}
\affil[ ]{\textit {\{calizo,chongben,wesleyg\}@pdx.edu}}

\date{\today}

\renewcommand\Affilfont{\normalsize\itshape}
\renewcommand\Authfont{\large}
\setlength{\affilsep}{6mm}
\renewcommand\Authsep{\rule{10mm}{0mm}}
\renewcommand\Authands{\rule{10mm}{0mm}}

\maketitle
\setcounter{page}{0}

\thispagestyle{empty}

\begin{abstract}

  An executive summary of your report.


  \vfill
  Your report should be about 10 pages without counting this title
  page or the references in the end. Give a comprehensive account but
  concisely.
\end{abstract}

\newpage

%------------------------------------------------------------------------------%
\section{Introduction}
\label{sec:intro}
%------------------------------------------------------------------------------%

The introduction is the most important part of a paper. A reader
should walk away with a clear picture of what you've done, a timeline
and transformative moments on this topic (which means references will
be crucial e.g.,~\cite{GM84}), why they are interesting and important,
what critical ideas there are, and what's next in this line of work.

\subsubsection*{Organization} The remainder of the paper is organized
as follows: $\ldots$

%------------------------------------------------------------------------------%
\section{Preliminaries}
\label{sec:prelim}
%------------------------------------------------------------------------------%
This section provides preliminary materials, such as summarizing
common notations, definitions, and useful facts for the main body.

\subsection{Notation}
We use the following notations in the discussion of our attack.

\begin{description}[style=unboxed, leftmargin=2em, font=\itshape]
    \item[Numbers:] To differentiate between hexadecimal, decimal, and binary number a subscript is used to denote the number's base. Decimal numbers are denoted with no subscript. (e.g., $31 =\text{1F}_{16} = 00011111_{2}$)
    
    \item[Plaintext and Ciphertext:] The plaintext is denoted by $M$, and the ciphertext is denoted by $C$. Because the initial permutation has little bearing on our attack model, $M$ represents the plaintext after the initial permutation, and $C$ represents the ciphertext before the final permutation.

    \item[Feistel Rounds:] The state of the ciphertext after round is $i$ denoted by the pair $(L_i, R_i)$, where $L_i$ denotes the leftmost 32-bits and $R_i$ denotes the rightmost 32-bits. The pair $(L_0, R_0)$ is used to denote the input to the first round of the feistel network.

    \item[Differences:] In differential cryptanalysis, variables are usually considered in pairs. Given a value $X$, its corresponding value in the pair is denoted by $X'$. The difference of $X$ and $X'$ is defined as $X \oplus X'$, and this value is denoted by $\Delta X$.

    \item[Subkeys:] The subkeys is denoted by $K_i$, where $i$ indicates the round in which the subkey is used.

    \item[Initial Permutation:] The initial permutation of DES is denoted by $IP(X)$, and the inverse initial permutation, or final permutation, is denoted by $IP^{-1}(X)$.
    
    \item[F Function:] The F function of DES is denoted by $F(X)$.
    
    \item[Expansion Function:] The expansion function is denoted by $E(X)$.

    \item[P Permutation:] The P permutation is denoted by $P(X)$.

    \item[S-boxes:] The eight S-boxes of DES are denoted by $S_1, S_2, \dots,S_8$. The 6-bit input to each S-box is denoted by $B_i$, where $i$ indicates its position. Similarly, the 4-bit output to each S-box is denoted by $O_i$.
    
\end{description}

%------------------------------------------------------------------------------%
\section{Main content}
\label{sec:result}
%------------------------------------------------------------------------------%

Pick appropriate titles and subtitles according to the topic of your
project.

%------------------------------------------------------------------------------%
\subsection{Result 1}
\label{sec:result1}
%------------------------------------------------------------------------------%

%------------------------------------------------------------------------------%
\subsection{Result 2}
\label{sec:result2}
%------------------------------------------------------------------------------%


%------------------------------------------------------------------------------%
\section{Something more}
\label{sec:more}
%------------------------------------------------------------------------------%


%-----------------------------------------------------------------------------%
\section{Conclusion}
\label{sec:con}
%-----------------------------------------------------------------------------%

Some concluding remarks. No need to repeat the content of the
report. Give some prospect on the topic of your choice. What are the
important take-away messages? Any open questions? 

%-----------------------------------------------------------------------------%
\subsection*{Acknowledgments}
%-----------------------------------------------------------------------------%

Optional. 

\newpage

\appendix

%------------------------------------------------------------------------------%
\section{Some supplementary material}
\label{sec:appendix}
%------------------------------------------------------------------------------%

Something you think is worth mentioning but not essential or digresses
the flow of the main body, e.g., long proofs of some claims.  Use it
scarcely. (Is it really worth including? Unless 100\% certain, the
answer is most likely no.)

% --------------------------------------------------------------------------- %
% References
% Check overleaf tutorial on managing refs with bibtex \url{https://www.overleaf.com/learn/latex/Bibliography_management_with_bibtex}
% --------------------------------------------------------------------------- %
\newpage
\bibliographystyle{alpha}
% \bibliographystyle{acm}

\bibliography{project.bib}

\end{document}

% LocalWords:  transformative

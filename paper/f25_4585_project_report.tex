\documentclass[11pt]{article}

\pdfoutput=1

% --------------------------------------------------------------------------- %
% Packages:
% --------------------------------------------------------------------------- %

\usepackage{amsmath,amsthm,amssymb,mathtools} %
\usepackage{authblk} %
\usepackage[english]{babel} %
\usepackage{cite} % references in numerical order
\usepackage{color} %
\usepackage{csquotes} %
\usepackage{dsfont} %
\usepackage[breaklinks]{hyperref} %
\usepackage[hmargin=1in,vmargin=1in]{geometry} %
\usepackage{graphicx} %
\usepackage{hyphenat} %
\usepackage{mathpazo} % charter, fourier, mathpazo, times
\usepackage{mdframed} %
\usepackage{suffix} % for *-version commands
\usepackage{tikz} %
\usepackage{xspace} %

% --------------------------------------------------------------------------- %
% Additional Packages not in template:
\usepackage{enumitem}
% --------------------------------------------------------------------------- %

\frenchspacing

\hypersetup{colorlinks=true, linkcolor=blue, citecolor=magenta}

\usetikzlibrary{calc,positioning}

\mdfdefinestyle{figstyle}{ %
  linecolor=black!7, %
  backgroundcolor=black!7, %
  innertopmargin=10pt, %
  innerleftmargin=25pt, %
  innerrightmargin=25pt, %
  innerbottommargin=10pt %
}

\definecolor{White}{rgb}{1,1,1} %
\definecolor{Black}{rgb}{0,0,0} %
\definecolor{LightGray}{rgb}{.8,.8,.8} %
\colorlet{ChannelColor}{LightGray} %
\colorlet{ChannelTextColor}{Black} %
\colorlet{ReadoutColor}{White} %

% --------------------------------------------------------------------------- %
% Macros:
% --------------------------------------------------------------------------- %

\newtheorem{theorem}{Theorem} %
\newtheorem{lemma}[theorem]{Lemma} %
\newtheorem{proposition}[theorem]{Proposition} %
\newtheorem{corollary}[theorem]{Corollary} %
\theoremstyle{definition} %
\newtheorem{definition}[theorem]{Definition} %
\theoremstyle{remark} %
\newtheorem{remark}{Remark} %

\newcommand{\comment}[1]{\begin{quote}\sf
    \textcolor{blue}{[#1]}\end{quote}} %

\newcommand{\complex}{\mathbb{C}} %
\newcommand{\real}{\mathbb{R}} %
\renewcommand{\natural}{\mathbb{N}} %
\renewcommand{\Re}{\operatorname{Re}} %
\renewcommand{\Im}{\operatorname{Im}} %
\newcommand{\secpar}{\kappa} %
\newcommand{\secparam}{1^\secpar} %
\newcommand{\poly}{\mathit{poly}} %
\def\yes{\text{yes}} %
\def\no{\text{no}} %

\newenvironment{mylist}[1]{\begin{list}{}{ %
      \setlength{\leftmargin}{#1} %
      \setlength{\rightmargin}{0mm} %
      \setlength{\labelsep}{2mm} %
      \setlength{\labelwidth}{8mm} %
      \setlength{\itemsep}{0mm}}}{\end{list}}

%% End-Of-Header

% --------------------------------------------------------------------------- %
% Your customization:
% --------------------------------------------------------------------------- %
\def\course{[\textsc{Fall 2025 CS485/585}]}

\newcommand{\notationdef}[2]{%
  \par\noindent
  \hangindent=2em
  \hangafter=1
  \textit{#1} #2\par\vspace{0.5em}
}

% --------------------------------------------------------------------------- %
% Main document:
% --------------------------------------------------------------------------- %

\begin{document}

\title{\course \\
  \LARGE\bf {Project report: Differential Cryptanalysis against 1-Round DES}}

\author[ ]{Jarren Calizo}
\author[ ]{Benjamin Chong}
\author[ ]{Wesley Grzemkowski}

\affil[ ]{Department of Computer Science\protect\\
  Portland State University\vspace{2mm}}
\affil[ ]{\textit {\{calizo,chongben,wesleyg\}@pdx.edu}}

\date{\today}

\renewcommand\Affilfont{\normalsize\itshape}
\renewcommand\Authfont{\large}
\setlength{\affilsep}{6mm}
\renewcommand\Authsep{\rule{10mm}{0mm}}
\renewcommand\Authands{\rule{10mm}{0mm}}

\maketitle
\setcounter{page}{0}

\thispagestyle{empty}

\begin{abstract}

  An executive summary of your report.


  \vfill
  Your report should be about 10 pages without counting this title
  page or the references in the end. Give a comprehensive account but
  concisely.
\end{abstract}

\newpage

%------------------------------------------------------------------------------%
\section{Introduction}
\label{sec:intro}
%------------------------------------------------------------------------------%

The introduction is the most important part of a paper. A reader
should walk away with a clear picture of what you've done, a timeline
and transformative moments on this topic (which means references will
be crucial e.g.,~\cite{GM84}), why they are interesting and important,
what critical ideas there are, and what's next in this line of work.

\subsubsection*{Organization} The remainder of the paper is organized
as follows: $\ldots$

%------------------------------------------------------------------------------%
\section{Preliminaries}
\label{sec:prelim}
%------------------------------------------------------------------------------%
This section provides preliminary materials, such as summarizing
common notations, definitions, and useful facts for the main body.

\subsection{Notation}
We use the following notations in the discussion of our attack.

\begin{description}[style=unboxed, leftmargin=2em, font=\itshape]
    \item[Numbers:] To differentiate between hexadecimal, decimal, and binary number a subscript is used to denote the number's base. Decimal numbers are denoted with no subscript. (e.g., $31 =\text{1F}_{16} = 00011111_{2}$)
    
    \item[Plaintext and Ciphertext:] The plaintext is denoted by $M$, and the ciphertext is denoted by $C$. Because the initial permutation has little bearing on our attack model, $M$ represents the plaintext after the initial permutation, and $C$ represents the ciphertext before the final permutation.

    \item[Feistel Rounds:] The state of the ciphertext after round is $i$ denoted by the pair $(L_i, R_i)$, where $L_i$ denotes the leftmost 32-bits and $R_i$ denotes the rightmost 32-bits. The pair $(L_0, R_0)$ is used to denote the input to the first round of the feistel network.

    \item[Differences:] In differential cryptanalysis, variables are usually considered in pairs. Given a value $X$, its corresponding value in the pair is denoted by $X'$. The difference of $X$ and $X'$ is defined as $X \oplus X'$, and this value is denoted by $\Delta X$. The notation $X^*$ is used to denote a characteristic, or expected differential, in $X$.

    \item[Subkeys:] The subkeys is denoted by $K_i$, where $i$ indicates the round in which the subkey is used. The notation $K^j$ is used to denote the $j$-th 6-bit chunk of $K$.

    \item[Initial Permutation:] The initial permutation of DES is denoted by $IP(X)$, and the inverse initial permutation, or final permutation, is denoted by $IP^{-1}(X)$.
    
    \item[F Function:] The F function, or mangler function, is denoted by $F(X)$.
    
    \item[Expansion Function:] The expansion function is denoted by $E(X)$.

    \item[P Permutation:] The P permutation is denoted by $P(X)$.

    \item[S-boxes:] The collective 48-bit to 32-bit S-box layer is denoted by the function $S(X)$. The individual eight S-boxes are denoted by $S^1, S^2, \dots,S^8$.

    \item[Intermediary Values:] The output of the expansion function is denoted by $B$. The input to the S-box layer is denoted by $I$. The output of the S-box layer is denoted by $O$. A superscript $j$ is used to denote the $j$-th 6-bit chunk, or 4-bit chunk.
    
\end{description}

\subsection{Differential Cryptanalysis}

%------------------------------------------------------------------------------%
\section{Description of Attack}
\label{sec:result}
%------------------------------------------------------------------------------%

Pick appropriate titles and subtitles according to the topic of your
project.

%------------------------------------------------------------------------------%
\subsection{The Attack Model}
\label{sec:result1}
%------------------------------------------------------------------------------%

The attack shown is a chosen-plaintext against 1-round DES.  The reduction to 1-round DES was done to demonstrate the impact of differential cryptanalysis on DES. This attack model is not intended to be a practical consideration. Our goal is to instead use 1-round DES to simulate a single round of more complex encryption algorithm, and we have adjusted our attack model accordingly.

To understand why, first note that 1-round DES is trivial to construct an attack against without any differential cryptanalysis. Consider a known plaintext-ciphertext pair $M = (L_0, R_0)$ and $C = (L_1, R_1)$. With this pair, an adversary computes $B = E(R_0)$ and $O = P^{-1}(R_1 \oplus L_0)$. The adversary then computes $B^j \oplus I^j = K^j$ for all possiblilities of $I^j$ and is left with only 4 potential values for $K^j$. After repeating this for $j = 1, \dots, 8$ the size of the key space is reduced to only $4^8 = 2^{16}$.

Now, consider an $n$-round DES scheme with a plaintext-cipertext pair $M = (L_0, R_0)$ and $C = (L_n, R_n)$. Note that $R_n = L_{n-1} \oplus F(L_n, k)$ and only $L_n$ and $R_n$ are known to the adversary. If the adversary knows $L_{n-1}$, they can learn $F(L_n, k)$ and use the attack previously described. This was only possible in 1-round DES because $L_0 = L_{n-1}$. Because of this, in our attack model we prohibit the adversary from using the actual value of $M$ after sending them to encryption oracle. How this exactly works will become more clear as we descibe our attack.

Finally, in our model, we consider the attack to be successful if it can output the first subkey $K_1$, with a high probability of correctness. In a complete DES scheme, this would be similar to recovering only the final subkey. We are justified in this approach because it is simple to derive bits in the actual key even from a single subkey. (TODO CITATION NEEDED)

%------------------------------------------------------------------------------%
\subsection{Choosing a Characteristic}
\label{sec:result2}
%------------------------------------------------------------------------------%

The goal of our attack is to generate a number ciphertext pairs $(C, C')$ where $\Delta I$ and $\Delta O$ are known to the adversary. Using such a ciphertext pair, an adversary can use $L_1$ to compute a subset of possible subkeys able to produce this result. Given a enough of these ciphertext pairs, an adversary can reduce the subkey space considerably, and it becomes possible to brute-force the exact subkey.

The general algorithm is as follows:
\begin{enumerate}
    \item Select a characteristic $(I^*, O^*)$ which occurs with high probability in $S$.
    \item Generate a set of plaintext pairs $\mathcal{M}$ where $\Delta B = I^*$.
    \item Encrypt the all pairs in $\mathcal{M}$ to get a set of ciphertext pairs $\mathcal{C}$.
    \item Derive a new set $\mathcal{C}^*$ which only contains pairs in $\mathcal{C}$ where $\Delta O = O^*$.
    \item For each $(C, C') \in \mathcal{C}^*$, compute all possible subkeys and add them to a set $\mathcal{K}^*$.
    \item The subkey $K$ will be one of the subkeys $\mathcal{K}^*$ which occurs with the highest frequency.
\end{enumerate}

For our attack, it is desireable for our characteristic $(I^*, O^*)$ when given a uniformly random $I$, $S(I) \oplus S(I') = O^*$ with the highest possible probability. This is because it will increase the likelihood for any given pair $(C, C') \in \mathcal{C}$ that $\Delta O = O^*$. This will allow us to produce a $\mathcal{C}^*$ of sufficient size with a reduced number of encryptions.

%------------------------------------------------------------------------------%
\subsection{Generating Plaintext Ciphertext Pairs}
\label{sec:result2}
%------------------------------------------------------------------------------%


%------------------------------------------------------------------------------%
\subsection{Reducing Key Space}
\label{sec:result2}
%------------------------------------------------------------------------------%


%------------------------------------------------------------------------------%
\subsection{Brute-forcing Subkey}
\label{sec:result2}
%------------------------------------------------------------------------------%

%------------------------------------------------------------------------------%
\section{Translating to Multi-Round DES}
\label{sec:more}
%------------------------------------------------------------------------------%


%-----------------------------------------------------------------------------%
\section{Conclusion}
\label{sec:con}
%-----------------------------------------------------------------------------%

This project implemented and experimentally evaluated a differential cryptanalysis attack on a one-round instance of DES, with the goal of recovering the last-round subkey from chosen plaintext-ciphertext pairs. By constructing S-box difference distribution tables, generating structured plaintext pairs with fixed input differences, and aggregating votes over candidate subkeys, the attack consistently reduced the 48-bit last-round key space from 2^48 possibilities down to roughly 2^10-2^12 candidates, after which a small brute-force search recovered the correct subkey. Although 1-round DES is algebraically trivial to invert given a single known plaintext-ciphertext pair, treating it as a target for differential cryptanalysis provided a concrete, code-level view of how biased differentials arise and how they can be exploited.

Our findings align qualitatively with the classical literature. Biham and Shamir's original work showed that full 16-round DES admits differential characteristics with non-negligible bias, but exploiting them requires about 2^47 chosen plaintexts and substantial time, pushing such attacks beyond realistic deployment scenarios even though they are asymptotically faster than exhaustive key search. Matsui's linear cryptanalysis paints a similar picture from a different angle: linear approximations can be used to recover DES keys with around 2^43-2^47 known plaintexts, again demonstrating structural weaknesses that are mostly of academic interest. Our 1-round experiments can be viewed as a scaled-down analogue of these results, where the same differential machinery becomes visible with far fewer pairs and the implementation details remain tractable for our cryptography class's final project.

At the same time, the limitations of our model are significant. We attack only the final round, not the full 16-round cipher; we assume ideal chosen-plaintext access to the encryption oracle; and we do not attempt to propagate differentials across multiple rounds or invert the full key schedule to recover the 56-bit master key. In practice, modern attackers would simply brute force single-DES or target Triple DES and legacy protocols, while contemporary standards have moved to AES and other ciphers with larger keys and stronger security margins. Our contribution is therefore best understood as a educational case study rather than a new break.

Nevertheless, the exercise is valuable. Implementing DES, building DDTs, and debugging the attack sharpened our understanding of how S-boxes, permutations, and key mixing interact under differential analysis. The fact that meaningful key-space reductions are achievable even in this simple setting reinforces why block cipher designers must select S-boxes and round counts with respect to differential and linear criteria, not just intuition. Future work could extend this framework to multiple rounds, compare DES's S-boxes against randomly generated alternatives, or replicate the experiment on modern lightweight ciphers to contrast their resistance profiles with DES.


%-----------------------------------------------------------------------------%
\subsection*{Acknowledgments}
%-----------------------------------------------------------------------------%

Optional. 

\newpage

\appendix

%------------------------------------------------------------------------------%
\section{Some supplementary material}
\label{sec:appendix}
%------------------------------------------------------------------------------%

Something you think is worth mentioning but not essential or digresses
the flow of the main body, e.g., long proofs of some claims.  Use it
scarcely. (Is it really worth including? Unless 100\% certain, the
answer is most likely no.)

% --------------------------------------------------------------------------- %
% References
% Check overleaf tutorial on managing refs with bibtex \url{https://www.overleaf.com/learn/latex/Bibliography_management_with_bibtex}
% --------------------------------------------------------------------------- %
\newpage
\bibliographystyle{alpha}
% \bibliographystyle{acm}

\bibliography{project.bib}

\end{document}

% LocalWords:  transformative

\documentclass[11pt]{article}

\pdfoutput=1

% --------------------------------------------------------------------------- %
% Packages:
% --------------------------------------------------------------------------- %

\usepackage{amsmath,amsthm,amssymb,mathtools} %
\usepackage{authblk} %
\usepackage[english]{babel} %
\usepackage{cite} % references in numerical order
\usepackage{color} %
\usepackage{csquotes} %
\usepackage{dsfont} %
\usepackage[breaklinks]{hyperref} %
\usepackage[hmargin=1in,vmargin=1in]{geometry} %
\usepackage{graphicx} %
\usepackage{hyphenat} %
\usepackage{mathpazo} % charter, fourier, mathpazo, times
\usepackage{mdframed} %
\usepackage{suffix} % for *-version commands
\usepackage{tikz} %
\usepackage{xspace} %
\usepackage{enumitem} % Added to handle notation section

\frenchspacing

\hypersetup{colorlinks=true, linkcolor=blue, citecolor=magenta}

\usetikzlibrary{calc,positioning}

\mdfdefinestyle{figstyle}{ %
  linecolor=black!7, %
  backgroundcolor=black!7, %
  innertopmargin=10pt, %
  innerleftmargin=25pt, %
  innerrightmargin=25pt, %
  innerbottommargin=10pt %
}

\definecolor{White}{rgb}{1,1,1} %
\definecolor{Black}{rgb}{0,0,0} %
\definecolor{LightGray}{rgb}{.8,.8,.8} %
\colorlet{ChannelColor}{LightGray} %
\colorlet{ChannelTextColor}{Black} %
\colorlet{ReadoutColor}{White} %

% --------------------------------------------------------------------------- %
% Macros:
% --------------------------------------------------------------------------- %

\newtheorem{theorem}{Theorem} %
\newtheorem{lemma}[theorem]{Lemma} %
\newtheorem{proposition}[theorem]{Proposition} %
\newtheorem{corollary}[theorem]{Corollary} %
\theoremstyle{definition} %
\newtheorem{definition}[theorem]{Definition} %
\theoremstyle{remark} %
\newtheorem{remark}{Remark} %

\newcommand{\comment}[1]{\begin{quote}\sf
    \textcolor{blue}{[#1]}\end{quote}} %

\newcommand{\complex}{\mathbb{C}} %
\newcommand{\real}{\mathbb{R}} %
\renewcommand{\natural}{\mathbb{N}} %
\renewcommand{\Re}{\operatorname{Re}} %
\renewcommand{\Im}{\operatorname{Im}} %
\newcommand{\secpar}{\kappa} %
\newcommand{\secparam}{1^\secpar} %
\newcommand{\poly}{\mathit{poly}} %
\def\yes{\text{yes}} %
\def\no{\text{no}} %

\newenvironment{mylist}[1]{\begin{list}{}{ %
      \setlength{\leftmargin}{#1} %
      \setlength{\rightmargin}{0mm} %
      \setlength{\labelsep}{2mm} %
      \setlength{\labelwidth}{8mm} %
      \setlength{\itemsep}{0mm}}}{\end{list}}

%% End-Of-Header

% --------------------------------------------------------------------------- %
% Your customization:
% --------------------------------------------------------------------------- %
\def\course{[\textsc{Fall 2025 CS485/585}]}

\newcommand{\notationdef}[2]{%
  \par\noindent
  \hangindent=2em
  \hangafter=1
  \textit{#1} #2\par\vspace{0.5em}
}

% --------------------------------------------------------------------------- %
% Main document:
% --------------------------------------------------------------------------- %

\begin{document}

\title{\course \\
  \LARGE\bf {Project report: Differential Cryptanalysis against 1-Round DES}}

\author[ ]{Jarren Calizo}
\author[ ]{Benjamin Chong}
\author[ ]{Wesley Grzemkowski}

\affil[ ]{Department of Computer Science\protect\\
  Portland State University\vspace{2mm}}
\affil[ ]{\textit {\{calizo,chongben,wesleyg\}@pdx.edu}}

\date{\today}

\renewcommand\Affilfont{\normalsize\itshape}
\renewcommand\Authfont{\large}
\setlength{\affilsep}{6mm}
\renewcommand\Authsep{\rule{10mm}{0mm}}
\renewcommand\Authands{\rule{10mm}{0mm}}

\maketitle
\setcounter{page}{0}

\thispagestyle{empty}

\begin{abstract}

  An executive summary of your report.


  \vfill
  Your report should be about 10 pages without counting this title
  page or the references in the end. Give a comprehensive account but
  concisely. 
\end{abstract}

\newpage

%------------------------------------------------------------------------------%
\section{Introduction}
\label{sec:intro}
%------------------------------------------------------------------------------%

        DES is a 16-round encryption scheme that operates on a 64-bit message using a 56-bit key.
    The full 16-round DES scheme has proven rather secure considering its limited key space. It has however fallen out of use since the relatively small key space makes the scheme vulnerable to brute-force attacks. From what we have found in our research, reducing the number of rounds DES uses makes it vulnerable to differential cryptanalysis attacks.
    
        In this report, we will show the attack we implemented to recover the key for a one-round 
    DES scheme. We will also show how such an attack can be used to recover the key for DES with a higher round count with relative efficiency.

\subsubsection*{Organization} The remainder of the paper is organized
as follows: $\ldots$

%------------------------------------------------------------------------------%
\section{Preliminaries}
\label{sec:prelim}
%------------------------------------------------------------------------------%
This section provides preliminary materials, such as summarizing
common notations, definitions, and useful facts for the main body.

\subsection{Notation}
We use the following notations in the discussion of our attack.

\begin{description}[style=unboxed, leftmargin=2em, font=\itshape]
    \item[Numbers:] To differentiate between hexadecimal, decimal, and binary number a subscript is used to denote the number's base. Decimal numbers are denoted with no subscript. (e.g., $31 =\text{1F}_{16} = 00011111_{2}$)
    
    \item[Plaintext and Ciphertext:] The plaintext is denoted by $M$, and the ciphertext is denoted by $C$. Because the initial permutation has little bearing on our attack model, $M$ represents the plaintext after the initial permutation, and $C$ represents the ciphertext before the final permutation.

    \item[Feistel Rounds:] The state of the ciphertext after round is $i$ denoted by the pair $(L_i, R_i)$, where $L_i$ denotes the leftmost 32-bits and $R_i$ denotes the rightmost 32-bits. The pair $(L_0, R_0)$ is used to denote the input to the first round of the feistel network.

    \item[Differences:] In differential cryptanalysis, variables are usually considered in pairs. Given a value $X$, its corresponding value in the pair is denoted by $X'$. The difference of $X$ and $X'$ is defined as $X \oplus X'$, and this value is denoted by $\Delta X$.

    \item[Subkeys:] The subkeys is denoted by $K_i$, where $i$ indicates the round in which the subkey is used. The notation $K^j$ is used to denote the $j$-th 6-bit chunk of $K$.

    \item[Initial Permutation:] The initial permutation of DES is denoted by $IP(X)$, and the inverse initial permutation, or final permutation, is denoted by $IP^{-1}(X)$.
    
    \item[F Function:] The F function, or mangler function, is denoted by $F(X)$.
    
    \item[Expansion Function:] The expansion function is denoted by $E(X)$.

    \item[P Permutation:] The P permutation is denoted by $P(X)$.

    \item[S-boxes:] The collective 48-bit to 32-bit S-box layer is denoted by the function $S(X)$. The individual eight S-boxes are denoted by $S^1, S^2, \dots,S^8$.

    \item[Intermediary Values:] The output of the expansion function is denoted by $B$. The input to the S-box layer is denoted by $I$. The output of the S-box layer is denoted by $O$. A superscript $j$ is used to denote the $j$-th 6-bit chunk, or 4-bit chunk.
    
\end{description}

\subsection{1-round DES}

We chose to construct our attack against 1-round DES. It is important to first clarify that 1-Round DES is already a trivial algorithm to defeat. To show this, consider a single known plaintext-ciphertext pair $M =(L_0, R_0)$ and $C=(L_1, R_1)$. From this values, one can compute $B = E(R_0)$ and $O = P^{-1}(R_1 \oplus L_0)$.

%------------------------------------------------------------------------------%
\section{The Attack}
\label{sec:result}
%------------------------------------------------------------------------------%

Pick appropriate titles and subtitles according to the topic of your
project.

%------------------------------------------------------------------------------%
\subsection{Our Attack Model}
\label{sec:result1}
%------------------------------------------------------------------------------%

The attack we have created is a chosen-plaintext attack against 1-Round DES. However it could also be adapted to be a known-plaintext attack. This attack is not intended to be an effective or practical attack on single-round DES, but rather demonstrates how DES may be susceptible to differential cryptanalysis using a simplified model.

The goal of this attack is to generate a number ciphertext pairs where the following information is known: the difference in the plaintexts, the difference in the ciphertexts, and the input to the F function.

%------------------------------------------------------------------------------%
\subsection{Finding Characteristics}
\label{sec:result2}
%------------------------------------------------------------------------------%


%------------------------------------------------------------------------------%
\subsection{Generating Plaintext Ciphertext Pairs}
\label{sec:result2}
%------------------------------------------------------------------------------%


%------------------------------------------------------------------------------%
\subsection{Reducing Key Space}
\label{sec:result2}
%------------------------------------------------------------------------------%


%------------------------------------------------------------------------------%
\subsection{Brute-forcing Subkey}
\label{sec:result2}
%------------------------------------------------------------------------------%

%------------------------------------------------------------------------------%
\section{Something more}
\label{sec:more}
%------------------------------------------------------------------------------%


%-----------------------------------------------------------------------------%
\section{Conclusion}
\label{sec:con}
%-----------------------------------------------------------------------------%

Some concluding remarks. No need to repeat the content of the
report. Give some prospect on the topic of your choice. What are the
important take-away messages? Any open questions? 

%-----------------------------------------------------------------------------%
\subsection*{Acknowledgments}
%-----------------------------------------------------------------------------%

Optional. 

\newpage

\appendix

%------------------------------------------------------------------------------%
\section{Some supplementary material}
\label{sec:appendix}
%------------------------------------------------------------------------------%

Something you think is worth mentioning but not essential or digresses
the flow of the main body, e.g., long proofs of some claims.  Use it
scarcely. (Is it really worth including? Unless 100\% certain, the
answer is most likely no.)

% --------------------------------------------------------------------------- %
% References
% Check overleaf tutorial on managing refs with bibtex \url{https://www.overleaf.com/learn/latex/Bibliography_management_with_bibtex}
% --------------------------------------------------------------------------- %
\newpage
\bibliographystyle{alpha}
% \bibliographystyle{acm}

\bibliography{project.bib}

\end{document}

% LocalWords:  transformative
